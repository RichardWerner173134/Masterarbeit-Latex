\chapter{Saga Pattern}

% TODO Referenzierung des ursprünglichen Papers - 1987

\section{Allgemeine Probleme verteilter Transaktionen bei lang bestehenden Transaktionen}

- "Long Lived Transactions"
- Zentrales Problem: Locks in der Datenbank

\section{Funktionsweise}



- Saga ist erfolgreich, wenn alle Ts erfolgreich ausgeführt werden -> Übergang von: Ausgangszustand -> Ausführung T1 -> Ausführung T2 -> Endzustand
- Saga ist fehlgeschlagen, wenn ein T fehlschlägt
- Ausführung der Cs
- Systemzustand ist danach immer noch konsistent
- Backward Recovery: Alle Ts, die ausgeführt wurden, werden durch Ausführung des entsprechenden Cs gerollbackt
- Forward Recovery: Einführung von Save Points zwischen den Ausführungen (zB T1-T5, Checkpoint nach T2 und Checkpoint nach T3)
- Fehler in T3 führt zu Rollback bis letztem Save Point S1 (entspricht Ausführung von C3, Zustand nach Ausführung nach T1 und T2)
- Wiederaufnehmen der Saga: Ausführung von T3 - T5
- Kompletter Rollback, falls es nicht geht (Backward Recovery)

\subsection{Ts und Cs}

\subsection{Forward Recovery}

\subsection{Backward Recovery}

\subsection{Saga Execution Component}

\subsection{Transaktionslog}



\section{Implementierungsstrategie}

\subsection{Orchestration}

\subsection{Choreographie}

\section{Abwägung - wann 2PC und wann Saga}

\section{Abwägung - Wann Orchestration und wann Choreographie}

