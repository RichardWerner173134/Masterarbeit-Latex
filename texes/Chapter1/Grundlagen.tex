\section{Grundlagen}

Um das in dieser Arbeit betrachtete Saga-Pattern zu verstehen, sollen zuerst einige Grundlagen erläutert werden. Besonders die im Titel der Arbeit enthaltenen Begriffe \textit{System} und \textit{Konsistenz} sollen in diesem Abschnitt erläutert werden. 

\subsection{System allgemein}
Ein System beschreibt einen abgegrenzten Bereich der objektiven Realität. Außerhalb dieses Bereichs liegt die Umgebung, die somit nicht zum System gehört. Zwischen des Systems und seiner Umgebung befindet sich der Systemrand. 

\subsection{System in der Softwareentwicklung}
In der Softwareentwicklung besteht ein System aus einer Menge miteinander interagierenden Softwarekomponenten. Diese Komponenten arbeiten an einem gemeinsamen Ziel. Neben der Software und deren Quellcode gehören auch Nutzerhandbücher, Tests, Bestandteile für die Instandhaltung sowie Spezifikationen und Konzepte zum System. 

\subsection{Zustand von Systemen}
Ein Softwaresystem befindet sich zu jedem Zeitpunkt in einem Zustand. Der Wechsel eines Zustands ist die Folge von Nutzerinteraktionen und festgelegten Routinen. Damit das System reibungslos funktionieren kann, darf es nur zwischen gültigen Zuständen wechseln. % TODO Was ist ein gültiger Zustand


\subsection{Transaktionen}
Ein Übergang von einem Zustand in den nächsten Zustand im Modell des Zustandsautomats ist eine einzige Operation. Häufig äußert sich eine solche Operation jedoch in einer Folge von auszuführenden Operationen. Eine solche Folge von Operationen wird als Transaktion bezeichnet. Teil einer Transaktion kann beispielsweise eine oder mehrere Datenbankoperationen oder der Aufruf einer externen Schnittstelle sein. Transaktionen können auf verschiedenen Ebenen betrachtet werden. Auf Datenbankebene werden Transaktionen anders umgesetzt als auf Ebene der Geschäftslogik. 

\subsection{ACID Eigenschaft von Transaktionen}
\textit{ACID} ist ein Akronym, welches häufig verwendet wird, um Eigenschaften auszudrücken, die eine Transaktion auf Datenbankebene aufweisen sollte. Die Buchstaben stehen für \textit{A}tomicity (Atomarität), \textit{C}onsistency (Konsistenz), \textit{I}solation (Isolation) und \textit{D}urability (Dauerhaftigkeit). 

\par Atomarität beschreibt hier das Alles-Oder-Nichts-Prinzip. Eine Transaktion, also eine Folge von Datenbankoperationen, muss entweder alle Operationen ausführen oder in einem Fehlerfall alle Änderungen auf den ursprünglichen Zustand zurückrollen. Als illustratives Beispiel soll das Szenario einer Überweisung dienen. Eine Überweisung besteht aus dem Erhöhen des Kontostands des Debitors und dem Verringern des Kontostands des Kreditors. Eine solche Transaktion kann fehlschlagen. Dabei ist sicherzustellen, dass die initialen Kontostände wiederhergestellt werden. Wird die Atomarität nicht gewährleistet, könnte sich die Gesamtsumme des im System enthaltenen Systems verändern, Geld geht also verloren oder Geld wird aus dem Nichts generiert. Das ist natürlich nicht akzeptabel. 

\par Konsistenz lässt sich mit widerspruchsfrei übersetzen. War das System vor Ausführung der Transaktion widerspruchsfrei, also konsistent, dann ist es das nach Ausführung der Transaktion ebenso. Um dies zu gewährleisten, sind verschiedene Mechanismen im Datenbanksystem im Einsatz. Dies werden im folgenden Abschnitt beschrieben.

\par Isolation stellt sicher, dass sich parallel in Ausführung befindliche Transaktionen nicht gegenseitig beeinflussen können. Das bedeutet, dass eine Transaktion Teile der Datenbasis sperren können muss. Eine solche Sperre führt jedoch zu blockierendem Verhalten und schränkt somit die nebenläufige Benutzung der Datenbank ein. 

\par Dauerhaftigkeit von Transaktionen muss gewährleistet sein, damit bei einem Ausfall Transaktionsdaten nicht verloren gehen. Um Dauerhaftigkeit zu erreichen, wird ein Transaktionslog gefährt, welches im Ausfall verwendet werden kann, um die verlorenen Transaktionen nachzuspielen. 

\subsection{Konsistenz auf Datenbankebene}

Auf Datenbankebene wird die Konsistenz durch drei Mechanismen sichergestellt: Bereichsintegrität, Entitätsintegrität, Referentielle Integrität und logische Konsistenz.

Die \textit{Bereichsintegrität} stellt lediglich sicher, dass die Werte einer Spalte in einem definierten Wertebereich befinden. 

Die \textit{Entitätsintegrität} beschreibt die Eindeutigkeit eines Datensatzes und somit Fähigkeit, diesen Datensatz mittels Primärschlüssel eindeutig identifizieren zu können.

Die \textit{Referentielle Integrität} gewährleistet, dass Relationen zwischen Datensätzen nicht auf nicht existierende Datensätze zeigen können. Ist ein Fremdschlüssel gesetzt, so muss der referenzierte Datensatz existieren. 

Die in der Datenbank enthaltenen Daten drücken einen Sachverhalt aus. Die semantische Bedeutung dieser Daten kann nicht durch das Datenbanksystem sichergestellt werden. Es liegt in der Verantwortung des Entwicklers, allen Nutzerinteraktionen mit dem System die korrekten Transaktionen zuzuordnen. Ein System ist \textit{logisch konsistent}, wenn kein logisch inkonsistenter Zustand erreichbar ist. 