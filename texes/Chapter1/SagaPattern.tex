\section{Saga Pattern}

\section{Was sind die Probleme, die in verteilten ACID Transaktionen auftreten?}
%TODO
Im Grundlagenkapitel wurden die mit dem ACID-Konsistenzprinzip einhergehenden Probleme dargestellt:
\begin{itemize}
	\item Blockierung der Ressourcen
	\item Viele Netzwerkaufrufe - Chattiness
	\item Enge Kopplung der Dienste - 
\end{itemize}
\section{Woher kommt es?}
Referenzierung des Papers 1987 Garcia

\section{Bestandteile des Musters}
\subsection{Aufteilung der Transaktion in lokale Transaktionen, Nacheinanderausführung}

\subsection{Cs und Ts - Kompensierungen}

\subsection{Betrachtung des Zustands nach Erfolg/Misserfolg}

\subsection{Recovery-Mechanismen}
\subsubsection{Forward Recovery}
\subsubsection{Backward Recovery}
\subsubsection{Save-Points}

\subsection{Ausprägungen des Patterns}
\subsubsection{Orchestration}


\subsubsection{Choreografie}

\section{Anwendungsgebiete des Patterns - Welche Usecases erlauben die Verwendung dieses Patterns? Welche nicht?}

\subsection{Langlebige Transaktionen - LLT}
\subsection{Bezug auf den Geschäftsprozess}
\subsection{Verteilte Systemlandschaft}
\subsection{Reaktion auf verschiedene Antwortmöglichkeiten in der Geschäftslogik}
\subsection{Fehlerfälle - Geschäftslogik und Ausfälle}
Hier soll der Unterschied zwischen Fehlern in der Geschäftslogik und Fehler aufgrund Ausfällen erläutert werden.

