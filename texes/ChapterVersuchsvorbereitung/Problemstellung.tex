\section{Problemstellung}

In \cref{chapter:sagapattern} wurde das Saga-Pattern als ein Implementierungsmuster vorgestellt, welches ermöglichen soll, die ACID-Anforderungen in einem verteilten System nachzubilden. 

Ein verteiltes System steht immer vor der Herausforderung von Netzwerkfehlern. Verwenden Transaktionsteilnehmer eines Systems die Request-Response-Kommunikation, so besteht immer die Möglichkeit, dass einzelne Nachrichten ihr Ziel nicht erreichen. Wenn eine solche Kommunikation deterministischer Natur ist, kann der Sender seinen Request ohne Gefahr wiederholen. Die im Saga-Pattern verwendeten lokalen Transaktion stellen jedoch keine deterministische Abfrage dar, sondern verfolgen das Ziel eines Zustandswechsels des Empfängers. Wiederholt der Sender seine Requests, führt dies zu ungewünschten Nebenwirkungen.

Es stellt sich die Frage, ob in Saga-Systemen, die Request-Response-Kommunikation verwenden, Konsistenz gewährleistet werden kann. 