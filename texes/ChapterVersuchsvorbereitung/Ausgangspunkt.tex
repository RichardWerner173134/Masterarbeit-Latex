\section{Ausgangspunkt}

Es soll ein Geschäftsvorgang mittels Saga-Pattern in einem Microservicesystem entworfen, implementiert und bewertet werden. Die gewählte Geschäftsvorgang soll als LLT aufgefasst werden und eine verteilte Transaktion abbilden. 

Es soll eine Implementierungsform des Saga-Patterns gewählt werden. In der Zielstellung wird ein System gefordert, welches Request-Response-Kommunikation verwendet. Somit kann die Verwendung von Messagingkomponenten ausgeschlossen werden. Die auf diese Komponenten ausgerichtete Implementierung per Choreografie wird deshalb nicht gewählt. 

Als Implementierungsform des Saga-Patterns wird für diesen Versuch die Orchestrierung gewählt. 