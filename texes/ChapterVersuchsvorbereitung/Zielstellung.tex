\section{Zielstellung} \label{sec:zielstellung}

Die folgenden Kapitel dienen dem Zweck, das Saga-Pattern hinsichtlich Systemkonsistenz zu untersuchen. Dabei wird davon ausgegangen, dass jegliche Kommunikation per Request-Response-Protokolle abläuft. 

\paragraph*{1}
Es ist die Frage zu beantworten, unter welchen Bedingungen ein Microservice-System, welches mittels Saga-Pattern implementiert wurde, eine LLT abbilden kann. 

\paragraph*{2}
Es sind Fehlerquellen und Fehlertypen zu identifizieren, die einen inkonsistenten Systemzustand verursachen können. Es sollen Lösungen im Rahmen des Saga-Patterns formuliert, implementiert und evaluiert werden. 

\paragraph*{3}
Es soll eine Antwort darauf gefunden werden, welche Kriterien eine Schnittstelle erfüllen muss, um an einer LLT teilnehmen zu können.

\paragraph*{4}
Es soll beantwortet werden, ob aufgrund Netzwerkpartitionen auftretende Fehler in die Fehlerbehandlung des Saga-Patterns integrierbar sind. 
