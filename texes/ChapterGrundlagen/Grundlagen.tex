%Um das in dieser Arbeit betrachtete Saga-Pattern zu verstehen, sollen zuerst einige Grundlagen erläutert werden. Besonders die im Titel der Arbeit enthaltenen Begriffe \textit{System} und \textit{Konsistenz} sollen in diesem Abschnitt erläutert werden. 
%
%\subsection{System allgemein}
%Ein System beschreibt einen abgegrenzten Bereich der objektiven Realität. Außerhalb dieses Bereichs liegt die Umgebung, die somit nicht zum System gehört. Zwischen des Systems und seiner Umgebung befindet sich der Systemrand. 
%
%\subsection{System in der Softwareentwicklung}
%In der Softwareentwicklung besteht ein System aus einer Menge miteinander interagierenden Softwarekomponenten. Diese Komponenten arbeiten an einem gemeinsamen Ziel. Neben der Software und deren Quellcode gehören auch Nutzerhandbücher, Tests, Bestandteile für die Instandhaltung sowie Spezifikationen und Konzepte zum System. 
%
%\subsection{Zustand von Systemen}
%Ein Softwaresystem befindet sich zu jedem Zeitpunkt in einem Zustand. Der Wechsel eines Zustands ist die Folge von Nutzerinteraktionen und festgelegten Routinen. Damit das System reibungslos funktionieren kann, darf es nur zwischen gültigen Zuständen wechseln. % TODO Was ist ein gültiger Zustand



