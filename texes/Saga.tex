\chapter{Saga-Pattern}

\section{Mögliche Titel}
\begin{itemize}
	\item Realisierung konsistenter Microservice-Systeme unter Verwendung von Sagas
	\item Realisierung konsistenter Microservice-Systeme unter Verwendung des Saga-Patterns
\end{itemize}

\section{Wieso gibts das überhaupt?}
\begin{itemize}
	\item Monolithische Anwendung hat keine Probleme mit Atomarität (lokale Transaktionen sind einfach, Datenbank unterstützt Atomare Operationen)
	\item Transaktionen in verteilten Systemen: Wie kann man Atomarität erreichen? Nicht atomare Folge von Schreibe- und Sendeoperationen führen zu inkonsistenten Systemzuständen
	\item 2-Phase-Commit ist die herkömmliche Lösung, bringt aber viele meist nicht akzeptable Probleme mit sich (Chattiness, blockierend!!, wenig Durchsatz!!)
\end{itemize}

\section{Saga-Pattern} \label{Saga-Pattern-Theorie}
\begin{itemize}
	\item Kompensations-Mechanismus für verteilte Transaktionen (Was passiert, wenn eine Operation fehlschlägt? Timeout?)
	\item Sicherstellung der Konsistenz des Systemzustands (Verteilte Transaktion sollte ein Übergang zwischen zwei konsistenten Zuständen sein)
	\item Orchestration und Choreographie
	\item Idempotenz
	\item Synchron und Asynchrone Verfahren
	\item ACID und BASE
\end{itemize}

\section{Design eines Microservice-Systems unter Verwendung des Saga-Patterns}
\begin{itemize}
	\item Verwenden der Punkte aus \ref{Saga-Pattern-Theorie}
	\item Verfolgen unterschiedlicher Ansätze (Orchestration und Choreographie)
	\item Bewertung und Vergleich der Designs
\end{itemize}

\section{Problemstellung}
\par Die Realisierung von Transaktionen in verteilten Systemen bringt viele Probleme mit sich. Das Saga-Pattern ist eine Möglichkeit, diese Probleme im Kontext von Microservices zu bewältigen. 

\par Im Rahmen der Masterarbeit soll das Saga-Pattern vorgestellt werden und die mit dem Systemdesign verbundenen Überlegungen anhand eines zu Beginn definierten Fallbeispiels erläutert werden. 

\par Es sind zwei Architekturen mit dazugehörigem Design unter Verwendung des Choreographie- und des Orchestrationsansatzes zu entwerfen und zu bewerten und zu vergleichen. Die Bewertung und der Vergleich sind vorrangig unter dem Blickwinkel der Systemkonsistenz durchzuführen. 