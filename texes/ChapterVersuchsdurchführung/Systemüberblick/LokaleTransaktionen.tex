\section{Lokale Transaktionen}
Die einzelnen lokalen Transaktionen sollen nun identifiziert werden. Im Folgenden wird beschrieben, durch welchen Teilnehmer die lokale Transaktion jeweils ausgeführt wird. 

\paragraph*{Initialisierung} \mbox{}\\
Die Bestellung wird im OrderService entgegengenommen. Die Bestellung wird initialisiert. Der Saga-Prozess wird erstellt. 

\paragraph*{Validierung der Produktdaten} \mbox{}\\
Die in der Bestellung enthaltenen Produktdaten müssen validiert werden. Dies geschieht in zwei Schritten:
\begin{enumerate}
	\item Abfrage aller in der Bestellung enthaltenen Produktdaten am ArticleService
	\item Validierung
\end{enumerate}

\paragraph*{Blockierug der Artikel} \mbox{}\\
Die Ware wird im StockService für andere Bestellungen blockiert. 

\paragraph*{Geldabbuchung} \mbox{}\\
Das Käuferkonto muss belastet werden. Dabei wird der Geldbetrag des Käuferkontos in der entsprechenden Instanz des BankService verringert.

\paragraph*{Geldzubuchung} \mbox{}\\
Das Händlerkonto bekommt den selben Geldbetrag gutgeschrieben. Dies geschieht ebenfalls in der entsprechenden Instanz des BankService. 

\paragraph*{Auslösung der Lieferung} \mbox{}\\
Die blockierten Artikel der Lieferung werden aus dem Lager entfernt und die Lieferung wird ausgelöst. Dies geschieht im StockService.

\paragraph*{Lieferabschluss} \mbox{}\\
Die Ware trifft beim Kunden ein. Die Lieferung wird durch den Lieferanten bestätigt. Diese Information erreicht den StockService.
