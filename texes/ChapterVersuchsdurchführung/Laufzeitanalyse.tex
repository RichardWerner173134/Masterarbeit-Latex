\section{Laufzeitanalyse}

Das primäre Werkzeug der Optimierung der Zustandsautomaten war die Einführung von Retries. Es ist zu erwarten, dass die Einführung von Retries die Laufzeit verlängert. Im Folgenden sind die Ergebnisse der Laufzeitanalyse per Boxplot dargestellt. Als Visualisierung wird ein modifizierter Boxplot verwendet. In jedem Diagramm sind das obere und untere Quartil, der Median, das Maximum und Minimum sowie Ausreißer ablesbar. Jeder Boxplot stellt die Werte eines DEAs dar. Aus Gründen der Übersicht wurden die langen Bezeichnungen der Zustandsautomaten durch folgende Nummerierung ersetzt:

\begin{center}
	\begin{tabular}[h]{|p{8cm}|p{3cm}|}
		\hline
		Zustandsautomat & Nummerierung \\ \hline
		SmBasic & 1 \\ \hline
		SmBasicSafeRetries & 2 \\ \hline
		SmBasicNetworkFailureUnlimitedRetries & 3 \\ \hline
		SmIdempotencyBackwardRecovery & 4 \\ \hline		
		SmIdempotencyForwardRecovers & 5 \\ \hline
	\end{tabular}
\end{center}
\FloatBarrier

Es ist erkennbar, dass die Laufzeit für alle Automaten sehr nah beieinander liegt. Da im ersten Szenario keine Netzwerkfehler auftauchen, sind die Werte sehr wenig gestreut. Die Ausreißer sind ebenfalls erkennbar.

\subsection{Testszenario1}

\begin{figure}[!htbp]
\begin{minipage}{.45\textwidth}
	
\begin{tikzpicture}
	\begin{axis} [
		boxplot/draw direction=y,
		%area legend,
		height=8.0cm, width=\linewidth,
		xmin=0,xmax=6,xtick={1,2,3,4,5},
		xtick style = {draw=none}, % Hide tick line
		enlarge x limits,
		ymin=0, ymax=100, ytick={0,20,...,100},
		ylabel = {Laufzeit [s]},
		ymajorgrids,
		ytick style = {draw=none}, % Hide tick line
		enlarge y limits,
		every axis plot/.append style={fill,fill opacity=0.5},
		every boxplot/.style={mark=x,every mark/.append style={mark size=1.0pt}}
		]
		% 
		\addplot+ [boxplot prepared={upper quartile=54.29, lower quartile=42.51, upper whisker=71.96, lower whisker=24.84, median=50.05, draw position=1}, ] coordinates {(1, 11.2)(1, 16.85)(1, 17.92)(1, 18.43)(1, 18.52)(1, 18.85)(1, 19.1)(1, 19.35)(1, 19.39)(1, 19.47)(1, 20.03)(1, 20.16)(1, 20.71)(1, 21.52)(1, 21.63)(1, 22.54)(1, 22.74)(1, 23.11)(1, 23.18)(1, 23.27)(1, 23.45)(1, 23.7)(1, 24.12)(1, 24.38)};
		\addplot+ [boxplot prepared={upper quartile=58.16, lower quartile=46.64, upper whisker=75.44, lower whisker=29.36, median=52.17, draw position=2}, ] coordinates {};
		\addplot+ [boxplot prepared={upper quartile=58.33, lower quartile=45.94, upper whisker=76.92, lower whisker=27.36, median=50.05, draw position=3}, ] coordinates {(3, 80.85)(3, 83.43)};
		\addplot+ [boxplot prepared={upper quartile=59.87, lower quartile=46.49, upper whisker=79.94, lower whisker=26.42, median=50.64, draw position=4}, ] coordinates {};
		\addplot+ [boxplot prepared={upper quartile=58.71, lower quartile=50.14, upper whisker=71.57, lower whisker=37.29, median=53.97, draw position=5}, ] coordinates {};
		
	\end{axis}
\end{tikzpicture}
	\caption{Boxplot FinishOrders in Szenario 1}
	\label{fig:boxplot_finishorders_scenario1}
\end{minipage}\hspace{\fill}%
\begin{minipage}{.45\textwidth}
	\begin{tikzpicture}
	\begin{axis} [
		boxplot/draw direction=y,
		%area legend,
		height=8.0cm, width=\linewidth,
		xmin=0,xmax=6,xtick={1,2,3,4,5},
		xtick style = {draw=none}, % Hide tick line
		enlarge x limits,
		ymin=0, ymax=140, ytick={0,20,...,140},
		ylabel = {Laufzeit [s]},
		ymajorgrids,
		ytick style = {draw=none}, % Hide tick line
		enlarge y limits,
		every axis plot/.append style={fill,fill opacity=0.5},
		every boxplot/.style={mark=x,every mark/.append style={mark size=1.0pt}}
		]
		% 
		\addplot+ [boxplot prepared={upper quartile=82.18, lower quartile=63.92, upper whisker=109.57, lower whisker=36.53, median=71.92, draw position=1}, ] coordinates {(1, 13.27)(1, 13.79)(1, 18.29)(1, 18.84)(1, 19.14)(1, 19.72)(1, 20.44)(1, 20.47)(1, 21.37)(1, 23.87)(1, 24.39)(1, 25.28)(1, 25.9)(1, 26.26)(1, 27.92)};
		\addplot+ [boxplot prepared={upper quartile=79.96, lower quartile=62.82, upper whisker=105.67, lower whisker=37.11, median=66.71, draw position=2}, ] coordinates {(2, 109.77)(2, 110.53)(2, 111.07)(2, 112.63)};
		\addplot+ [boxplot prepared={upper quartile=82.83, lower quartile=66.65, upper whisker=107.1, lower whisker=42.38, median=71.76, draw position=3}, ] coordinates {(3, 109.05)(3, 112.61)};
		\addplot+ [boxplot prepared={upper quartile=71.11, lower quartile=61.99, upper whisker=84.79, lower whisker=48.31, median=67.06, draw position=4}, ] coordinates {(4, 45.01)(4, 45.23)(4, 46.59)(4, 47.09)(4, 86.45)(4, 86.52)(4, 87.74)(4, 90.29)(4, 93.78)(4, 94.11)};
		\addplot+ [boxplot prepared={upper quartile=86.44, lower quartile=64.11, upper whisker=119.94, lower whisker=30.62, median=77.08, draw position=5}, ] coordinates {(5, 121.27)(5, 128.37)};
		
		
	\end{axis}
\end{tikzpicture}
	\caption{Boxplot CancelOrders in Szenario 1}
	\label{fig:boxplot_cancelorders_scenario1}
\end{minipage}
\end{figure}
\FloatBarrier

\subsection{Testszenario2}

\begin{figure}[!htbp]
\begin{minipage}{.45\textwidth}
	\begin{tikzpicture}
	\begin{axis} [
		boxplot/draw direction=y,
		%area legend,
		height=8.0cm, width=\linewidth,
		xmin=0,xmax=6,xtick={1,2,3,4,5},
		xtick style = {draw=none}, % Hide tick line
		enlarge x limits,
		ymin=0, ymax=160, ytick={0,20,...,160},
		ylabel = {Laufzeit [s]},
		ymajorgrids,
		ytick style = {draw=none}, % Hide tick line
		enlarge y limits,
		every axis plot/.append style={fill,fill opacity=0.5},
		every boxplot/.style={mark=x,every mark/.append style={mark size=1.0pt}}
		]
		% 
		\addplot+ [boxplot prepared={upper quartile=52.08, lower quartile=5.19, upper whisker=122.42, lower whisker=0.0, median=22.41, draw position=1}, ] coordinates {};
		\addplot+ [boxplot prepared={upper quartile=56.52, lower quartile=23.91, upper whisker=105.44, lower whisker=0.0, median=47.41, draw position=2}, ] coordinates {};
		\addplot+ [boxplot prepared={upper quartile=62.87, lower quartile=52.21, upper whisker=78.86, lower whisker=36.22, median=56.77, draw position=3}, ] coordinates {(3, 79.24)(3, 80.46)(3, 84.34)(3, 87.85)(3, 89.73)(3, 90.99)(3, 91.46)(3, 94.78)(3, 96.65)(3, 97.25)(3, 97.69)(3, 101.48)(3, 111.56)};
		\addplot+ [boxplot prepared={upper quartile=51.25, lower quartile=29.41, upper whisker=84.01, lower whisker=0.0, median=41.6, draw position=4}, ] coordinates {(4, 87.31)};
		\addplot+ [boxplot prepared={upper quartile=61.88, lower quartile=46.44, upper whisker=85.04, lower whisker=23.28, median=53.52, draw position=5}, ] coordinates {(5, 86.24)(5, 87.15)(5, 90.83)(5, 101.85)(5, 106.66)(5, 115.02)};
		
	\end{axis}
\end{tikzpicture}
	\caption{Boxplot FinishOrders in Szenario 2}
	\label{fig:boxplot_finishorders_scenario2}
\end{minipage}\hspace{\fill}%
\begin{minipage}{.45\textwidth}
	\begin{tikzpicture}
	\begin{axis} [
	 	boxplot/draw direction=y,
	 	%area legend,
		height=8.0cm, width=\linewidth,
	 	xmin=0,xmax=6,xtick={1,2,3,4,5},
	 	xtick style = {draw=none}, % Hide tick line
	 	enlarge x limits,
	 	ymin=0, ymax=350, ytick={0,50,...,350},
	 	ylabel = {Laufzeit [s]},
	 	ymajorgrids,
	 	ytick style = {draw=none}, % Hide tick line
	 	enlarge y limits,
	 	every axis plot/.append style={fill,fill opacity=0.5},
	 	every boxplot/.style={mark=x,every mark/.append style={mark size=1.0pt}}
	 	]
	 	% 
		\addplot+ [boxplot prepared={upper quartile=60.61, lower quartile=3.33, upper whisker=146.53, lower whisker=0.0, median=17.97, draw position=1}, ] coordinates {};
		\addplot+ [boxplot prepared={upper quartile=67.01, lower quartile=15.64, upper whisker=144.07, lower whisker=0.0, median=60.54, draw position=2}, ] coordinates {};
		\addplot+ [boxplot prepared={upper quartile=86.73, lower quartile=67.71, upper whisker=115.26, lower whisker=39.18, median=76.62, draw position=3}, ] coordinates {(3, 116.89)(3, 119.51)(3, 127.43)};
		\addplot+ [boxplot prepared={upper quartile=70.15, lower quartile=34.93, upper whisker=122.98, lower whisker=0.0, median=63.66, draw position=4}, ] coordinates {};
		\addplot+ [boxplot prepared={upper quartile=83.82, lower quartile=65.59, upper whisker=111.17, lower whisker=38.25, median=75.02, draw position=5}, ] coordinates {};
		
		
	\end{axis}
\end{tikzpicture}
	\caption{Boxplot CancelOrders in Szenario 2}
	\label{fig:boxplot_cancelorders_scenario2}
\end{minipage}
\end{figure}
\FloatBarrier

\subsection{Testszenario3}

\begin{figure}[!htbp]
\begin{minipage}{.45\textwidth}
	\begin{tikzpicture}
	\begin{axis} [
		boxplot/draw direction=y,
		%area legend,
		height=8.0cm, width=\linewidth,
		xmin=0,xmax=6,xtick={1,2,3,4,5},
		xtick style = {draw=none}, % Hide tick line
		enlarge x limits,
		ymin=0, ymax=500, ytick={0,50,...,500},
		ylabel = {Laufzeit [s]},
		ymajorgrids,
		ytick style = {draw=none}, % Hide tick line
		enlarge y limits,
		every axis plot/.append style={fill,fill opacity=0.5},
		every boxplot/.style={mark=x,every mark/.append style={mark size=1.0pt}}
		]
		% 
		\addplot+ [boxplot prepared={upper quartile=41.04, lower quartile=2.74, upper whisker=98.49, lower whisker=0.0, median=30.93, draw position=1}, ] coordinates {(1, 101.82)(1, 102.95)(1, 108.75)(1, 108.95)};
		\addplot+ [boxplot prepared={upper quartile=66.71, lower quartile=29.92, upper whisker=121.9, lower whisker=0.0, median=45.8, draw position=2}, ] coordinates {(2, 129.58)(2, 130.59)(2, 143.72)(2, 154.37)(2, 182.7)(2, 203.3)(2, 243.21)(2, 244.16)(2, 248.2)(2, 254.98)(2, 309.51)(2, 344.46)(2, 432.65)(2, 440.62)};
		\addplot+ [boxplot prepared={upper quartile=113.35, lower quartile=54.57, upper whisker=201.52, lower whisker=0.0, median=77.61, draw position=3}, ] coordinates {(3, 204.14)(3, 208.81)(3, 212.37)(3, 221.45)(3, 224.64)(3, 244.14)(3, 244.85)(3, 247.33)(3, 250.85)(3, 259.94)(3, 260.75)(3, 272.22)(3, 275.4)(3, 288.21)(3, 320.91)(3, 349.6)(3, 450.19)(3, 648.61)(3, 702.9)(3, 732.25)(3, 1618.39)};
		\addplot+ [boxplot prepared={upper quartile=76.06, lower quartile=39.87, upper whisker=130.35, lower whisker=0.0, median=53.59, draw position=4}, ] coordinates {(4, 136.77)(4, 137.4)(4, 137.92)(4, 141.51)(4, 150.34)(4, 154.35)(4, 194.8)};
		\addplot+ [boxplot prepared={upper quartile=115.15, lower quartile=55.41, upper whisker=204.76, lower whisker=0.0, median=80.81, draw position=5}, ] coordinates {(5, 207.89)(5, 215.69)(5, 223.47)(5, 237.18)(5, 242.56)(5, 243.06)(5, 258.77)(5, 261.44)(5, 272.91)(5, 289.96)(5, 296.15)(5, 319.09)(5, 328.18)(5, 336.99)(5, 408.06)(5, 411.49)(5, 412.02)(5, 577.31)(5, 1045.85)};
		
	\end{axis}
\end{tikzpicture}
	\caption{Boxplot FinishOrders in Szenario 3}
	\label{fig:boxplot_finishorders_scenario3}
\end{minipage}\hspace{\fill}%
\begin{minipage}{.45\textwidth}
	\begin{tikzpicture}
	\begin{axis} [
		boxplot/draw direction=y,
		%area legend,
		height=8.0cm, width=\linewidth,
		xmin=0,xmax=6,xtick={1,2,3,4,5},
		xtick style = {draw=none}, % Hide tick line
		enlarge x limits,
		ymin=0, ymax=800, ytick={0,100,...,800},
		ylabel = {Laufzeit [s]},
		ymajorgrids,
		ytick style = {draw=none}, % Hide tick line
		enlarge y limits,
		every axis plot/.append style={fill,fill opacity=0.5},
		every boxplot/.style={mark=x,every mark/.append style={mark size=1.0pt}}
		]
		% 
		\addplot+ [boxplot prepared={upper quartile=43.64, lower quartile=5.47, upper whisker=100.9, lower whisker=0.0, median=31.02, draw position=1}, ] coordinates {(1, 109.38)(1, 110.29)(1, 110.77)};
		\addplot+ [boxplot prepared={upper quartile=86.18, lower quartile=33.64, upper whisker=164.99, lower whisker=0.0, median=53.31, draw position=2}, ] coordinates {(2, 174.58)(2, 175.32)(2, 203.06)(2, 213.26)(2, 263.81)(2, 315.44)(2, 389.57)(2, 410.06)};
		\addplot+ [boxplot prepared={upper quartile=156.65, lower quartile=88.51, upper whisker=258.86, lower whisker=0.0, median=114.01, draw position=3}, ] coordinates {(3, 263.86)(3, 268.53)(3, 270.02)(3, 313.86)(3, 320.18)(3, 355.59)(3, 361.24)(3, 362.67)(3, 400.73)(3, 527.3)(3, 625.53)(3, 802.82)};
		\addplot+ [boxplot prepared={upper quartile=215.92, lower quartile=73.04, upper whisker=430.24, lower whisker=0.0, median=127.89, draw position=4}, ] coordinates {};
		\addplot+ [boxplot prepared={upper quartile=152.41, lower quartile=84.71, upper whisker=253.96, lower whisker=0.0, median=113.1, draw position=5}, ] coordinates {(5, 308.42)(5, 313.7)(5, 375.55)(5, 390.3)(5, 392.51)(5, 427.96)(5, 428.06)(5, 563.98)(5, 620.34)};
		
	\end{axis}
\end{tikzpicture}
	\caption{Boxplot CancelOrders in Szenario 3}
	\label{fig:boxplot_cancelorders_scenario3}
\end{minipage}
\end{figure}





