\chapter{Einleitung}

\section{Motivation}

Wieso, Weshalb, Warum?

\section{Zielsetzung dieser Arbeit}

These: Mittels Saga-Pattern implementierte langlebige Transaktionen (LLT) ermöglichen Konsistenz in Microservicesystemen, die per Request-Response Pattern kommunizieren. 

Leitfragen: 
Wie kann mithilfe von BASE-Eigenschaft und Softstate ausfallsichere Konsistenz hergestellt werden?

(Wie) kann der 2 Phasencommit in eine Saga-Architektur integriert werden?

\section{Aufbau dieser Arbeit}

\begin{itemize}
	\item Kapitel 1: Theoretische Grundlagen: In diesem Kapitel sollen dem Leser die Grundlagen vermittelt werden, die er benötigt, um den Rest der Arbeit zu verstehen. Dazu gehören Begriffe und Konzepte aus der Welt der Microservices. 
	\item Kapitel 2: Methodik: Ziel des Kapitels: Entwurf, Implementierung, Messen und Bewertung des Systems. Fokus auf Saga-Pattern, Konsistenz, Ausfallsicherheit -> Der Systementwurf und der Bewertungsprozess aus diesem Kapitel sollen in Kapitel 3 umgesetzt werden.
	\begin{itemize}
		\item Bewertungskriterien
		\item Entwurf und Implementierung eines reinen Saga-Systems
		\item Abänderung der Implementierung durch 
		\begin{itemize}
			\item Saga-Werkzeuge (Checkpoints, Recovery Mechanismen)
			\item 2 Phasencommit
		\end{itemize}
	\end{itemize}
	\item Kapitel 3: Ergebnisse aus Kapitel 2: Die Ergebnisse dieses Kapitels sollen in Kapitel 4 verwendet werden, um die anfängliche These und die Leitfragen zu beantworten.
	\item Kapitel 4: Diskussion - Bewerten der Ergebnisse in Bezug auf die These und der Leitfragen; Beantwortung der These und der Leitfragen
	
\end{itemize}













