\section{Zielstellung}

Der praktische Teil dieser Arbeit soll dem Ziel dienen, die These zu beantworten:
% TODO These kopieren
(HIER WIRD DIE THESE WIEDERHOLT)

Dazu soll im ersten Schritt ein Microservicesystem  nach Vorbild des Saga-Patterns entworfen, entwickelt und bewertet werden. Der zu verwendende Prozess soll einen Geschäftsprozess abbilden, der für die Umsetzung mittels Saga-Pattern geeignet ist. Der Prozess soll also:
\begin{itemize}
	\item eine LLT darstellen,
	\item im Kontext einer verteilten Microservicearchitektur umgesetzt werden,
	\item auf verschiedene Fälle in der Geschäftslogik reagieren können und
	\item die ACID-Eigenschaften der LLT erfüllen.
\end{itemize}

Der zweite Schritt des praktischen Teils soll die ursprüngliche Implementierung abändern, sodass neben allen möglichen Fehlern, die auf Geschäftslogik zurückzuführen sind, auch auf Fehler reagiert werden kann, die auf Netzwerkausfälle zurückzuführen sind. 