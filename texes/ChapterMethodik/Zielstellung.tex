\section{Zielstellung}

Der praktische Teil dieser Arbeit soll dem Ziel dienen, die These zu beantworten:
% TODO These kopieren
(HIER WIRD DIE THESE WIEDERHOLT)

Dazu soll im ersten Schritt ein Microservicesystem  nach Vorbild des Saga-Patterns entworfen, entwickelt und bewertet werden. Der zu verwendende Prozess soll einen Geschäftsprozess abbilden, der für die Umsetzung mittels Saga-Pattern geeignet ist. Der Prozess soll also:
\begin{itemize}
	\item LLTs enthalten,
	\item im Kontext eine verteilten Microservicearchitektur umgesetzt werden können und
	\item auf verschiedene Fälle in der Geschäftslogik reagieren können.
\end{itemize}

Die Mindestvoraussetzung ist, dass das umgesetzte System die genannten Kriterien erfüllt. Neben der Reaktion auf Fehler in der Geschäftslogik soll untersucht werden, wie das System auf Ausfälle reagiert. 

Der zweite Schritt des praktischen Teils soll die Kritikpunkte der ersten Implementierung verwenden und Lösungen für potentielle Probleme liefern. 