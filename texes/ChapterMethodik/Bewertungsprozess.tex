\section{Bewertungskriterien}

In diesem Abschnitt werden die verschiedenen Bewertungskriterien für das zu entwerfende und zu implementierende System festgelegt. 

konkreter Bewertungsprozess mit Punkteverteilung folgt. %TODO

\subsection{Anzahl der Pfade}
ganzzahlige Werte und Prozentwerte im Vergleich zur Gesamtanzahl der Programmpfade


Es soll die Anzahl an Pfaden gezählt werden, die die Saga koordinieren. 

\begin{itemize}
	\item Gesamtanzahl der Programmpfade
	\item Anzahl der Programmpfade, die das System in einem konsistenten Zustand hinterlassen
		\begin{itemize}
			\item Anzahl der Programmpfade, die in einer erfolgreichen Saga resultieren
			\item Anzahl der Programmpfade, die in einer erfolglosen Saga resultieren
		\end{itemize}
	\item Anzahl der Programmpfade, die das System in einem inkonsistenten Zustand hinterlassen
\end{itemize}


\subsection{Anzahl der Netzwerkaufrufe}
ganzzahliger Wert

\begin{itemize}
	\item im erfolgreichen Pfad
	\item im erfolglosen Pfad mit Kompensierung
\end{itemize}

\subsection{Wahrscheinlichkeit für Ausfall}
Wahrscheinlichkeit

Jeder am Prozess teilhabende Microservice läuft auf einem Server. Um die Gesamtausfallwahrscheinlichkeit des Systems zu messen wird von einer konstanten Ausfallwahrscheinlichkeit pro Service ausgegangen. Die Gesamtausfallwahrscheinlichkeit ist die Summe der Wahrscheinlichkeiten, in denen das System einen Programmpfad verwendet, der die Daten in einem inkonsistenten Zustand hinterlässt.

\subsection{Konsistenzerhaltung im Falle eines Ausfalls der Nichtkoordinierenden Services}
ja oder nein

Es soll betrachtet werden, ob sich das System in einen konsistenten Zustand zurückfinden kann, falls einer der Services ausfällt, die vom Koordinator zu einer lokalen Transaktion aufgefordert werden können. 


\subsection{Konsistenzerhaltung im Falle eines Koordinatorausfalls}
ja oder nein 

Es soll betrachtet werden, ob sich das System in einen konsistenten Zustand zurückfinden kann, falls der koordinierende Service ausfällt. 