\chapter{Einleitung}

\section{Motivation}

Wieso, Weshalb, Warum?

\section{Zielsetzung dieser Arbeit}

These: Mittels Saga-Pattern implementierte lang andauernde Transaktionen (LLT) ermöglichen (k)eine ausfallsichere Konsistenz in Microservicesystemen. 

Leitfragen: 
Wie kann mithilfe von BASE-Eigenschaft und Softstate ausfallsichere Konsistenz hergestellt werden?

(Wie) kann der 2 Phasencommit in eine Saga-Architektur integriert werden?

\section{Aufbau dieser Arbeit}

\begin{itemize}
	\item Kapitel 1: Theoretische Grundlagen
	\item Kapitel 2: Methodik
	\begin{itemize}
		\item Entwurf und Implementierung eines reinen Saga-Systems
		\item Abänderung der Implementierung durch 
		\begin{itemize}
			\item Saga-Werkzeuge (Checkpoints, Recovery Mechanismen)
			\item 2 Phasencommit
		\end{itemize}
		\item Bewertung der Systeme
		\begin{itemize}
			\item Wie Fehleranfällig ist das System?
			\begin{itemize}
				\item Welche Single-Point-of-Failures gibt es?
				\item Welche Recovery-Mechanismen gibt es?
				\item Welche Datenanomalien sind die Folge eines Ausfalls?
			\end{itemize}
			\item Wie komplex ist das System?
			\item Wie hoch ist der Netzwerkverkehr? (Anzahl der benötigten Requests)
			\begin{itemize}
				\item Wie performant ist der Happy-Path? 
				\item Wie performant ist der Sad-Path?	
			\end{itemize}
		\end{itemize}
		\item Erläuterung des Mess- und Bewertungsvorgangs
	\end{itemize}
	\item Kapitel 3: Ergebnisse aus Kapitel 2
	\item Kapitel 4: Diskussion - Bewerten der Ergebnisse in Bezug auf die These und der Leitfragen; Beantwortung der These und der Leitfragen
	
\end{itemize}













